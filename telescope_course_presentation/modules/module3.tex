
\subsection{Equações}

\begin{frame}
	
	Uma equação como $y = a x^2 + b x + c$ pode ser inserida ao longo do texto de um parágrafo usando o ambiente \LaTeX\ \texttt{math} (ou o atalho \LaTeX\ \texttt{\textbackslash(\ldots\textbackslash)} ou o atalho \TeX\ \texttt{\$\ldots\$}) e calculada como $y = \fpeval{1 * 2^2 + 2 * 2 + 4}$ para $a = 1$, $b = 2$, $c = 4$ e $x = 2$.
	Por outro lado, a seguinte equação (não numerada) pode ser inserida em uma linha própria usando o ambiente \LaTeX\ \texttt{displaymath} (ou o atalho \LaTeX\ \texttt{\textbackslash[\ldots\textbackslash]}):
	\[
	\only<beamer:1>{\frac{\mathrm{d}y}{\mathrm{d}x} = \gamma \operatorname{sen} x}
	\only<all:0|beamer:2>{%
		\frac{\mathrm{d}y}{\mathrm{d}x} = \gamma \operatorname{sen} x%
		\rlap{\alert{ $\leftarrow$ integrando esta equação}}%
	}
	\only<all:0|beamer:3>{%
		y = - \gamma \cos x + C\vphantom{\frac{\mathrm{d}}{\mathrm{d}}}%
		\rlap{\alert{ $\leftarrow$ resulta nesta equação}}%
	}
	\only<all:0|beamer:4>{%
		y = \fpeval{- 4 * round(cos(pi / 3), 3) + 4}\vphantom{\frac{\mathrm{d}}{\mathrm{d}}}%% CHKTEX 35
		\rlap{\alert{ $\leftarrow$ para ${\gamma} = 4$, $C = 4$ e $x = \frac{\pi}{3}$}}%
	}
	\]
	
	A Equação~\eqref{eq:fx} foi inserida usando o ambiente \LaTeX\ \texttt{equation} e numerada automaticamente:
	\begin{equation}%
		\label{eq:fx}
		f(x) = \frac{1}{\alpha} \int_0^L \left(\frac{x^2}{2} - \frac{x^3}{3}\right) \mathrm{d}x
	\end{equation}
	
	\begin{alertblock}{\faInfoCircle\ Ferramentas para gerar ou editar equações em \LaTeX}
		
		\begin{itemize}
			\selectlanguage{english}
			\item[\tiny\faTools] \href{https://formulasheet.com/}{Formula Sheet\LinkIcon}.
			\item[\tiny\faTools] \href{https://www.tutorialspoint.com/latex_equation_editor.htm}{\LaTeX\ Equation Editor (\textit{by} Tutorials Point)\LinkIcon}.
		\end{itemize}
		
	\end{alertblock}
	
\end{frame}
