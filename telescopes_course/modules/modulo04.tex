\section{Opciones Avanzadas en SolidWorks}

\textbf{Objetivo:} Explorar funcionalidades avanzadas en SolidWorks como chapa metálica, planos para corte láser, simulación de partículas, análisis por elementos finitos y animaciones. Sólo la parte de corte láser. 


\section{Chapa Metálica (Sheet Metal)}

\subsection{Introducción a chapa metálica}
% Introducción general para que comprendan cuándo usar esta herramienta.
% Puedes incluir ejemplos como cajas, soportes o envolventes metálicos.
%\begin{itemize}
%	\item Conceptos básicos de chapa metálica.
%	\item Aplicaciones comunes en fabricación.
%\end{itemize}

\subsection{Operaciones básicas en chapa metálica}
% Operaciones clave que diferencian este módulo del modelado tradicional.
% Ideal para mostrar plegados y su comportamiento realista.
%\begin{itemize}
%	\item Brida base (Base Flange).
%	\item Pliegues (Edge Flange), pestañas y dobladillos.
%	\item Corte en chapa metálica.
%\end{itemize}

\subsection{Desplegado y preparación para fabricación}
% Este es el objetivo principal del módulo: obtener patrones planos para cortar y doblar.
%\begin{itemize}
%	\item Generar patrón plano (Flat Pattern).
%	\item Parámetros críticos (factor K, radios de doblado).
%\end{itemize}

\subsection{Exportación para corte láser o CNC}
% Muy útil si tus estudiantes o tú trabajan con fábricas, makerspaces o talleres.
%\begin{itemize}
%	\item Exportación en DXF/DWG.
%	\item Consejos para evitar errores al abrir en software de corte.
%\end{itemize}

\section{Creación de Planos para Corte Láser}

\subsection{Preparación del modelo}
% Aunque se puede exportar directamente un sketch, es mejor preparar bien el archivo.
%\begin{itemize}
%	\item Consideraciones geométricas: líneas cerradas, espesor constante.
%	\item Tolerancias adecuadas para encajes si se va a ensamblar.
%\end{itemize}

\subsection{Generación de planos específicos para corte/armado}
% Puedes enseñar a generar el dibujo técnico, acotar, y luego exportar solo las vistas deseadas.
%\begin{itemize}
%	\item Vistas planas del modelo.
%	\item Escalas correctas, capas para corte y grabado.
%\end{itemize}

\subsection{Exportación en formatos compatibles}
% Esta parte es crítica si se va a llevar el archivo a una cortadora láser real.
%\begin{itemize}
%	\item Formato DXF (más común) y DWG.
%	\item Verificar visualmente en otros programas como LibreCAD o Inkscape.
%\end{itemize}

\section{Simulación de Partículas (Flow Simulation)}

\subsection{Conceptos básicos}
% Aquí se introduce la idea de fluidos computacionales (CFD) en el entorno de SolidWorks.
% Puedes usar ejemplos simples como ventilación, flujo de aire o agua.
%\begin{itemize}
%	\item ¿Qué es CFD?
%	\item Ejemplos sencillos: flujo de aire en una carcasa, enfriamiento.
%\end{itemize}

%\subsection{Creación de estudios de flujo}
% Es útil que vean cómo definir materiales, entrada y salida del flujo.
%\begin{itemize}
%	\item Configuración inicial del dominio.
%	\item Entrada, salida, presión y temperatura.
%\end{itemize}

\subsection{Interpretación básica de resultados}
% Aquí puedes mostrar cómo visualizar trayectorias de flujo y campos de presión.
%\begin{itemize}
%	\item Trayectorias de partículas.
%	\item Campos de velocidad y presión.
%\end{itemize}

\section{Análisis por Elementos Finitos (FEA)}

\subsection{Conceptos iniciales}
% Breve introducción a FEA. Aclara que es una simulación estática simple.
%\begin{itemize}
%	\item Definición de FEA y su utilidad.
%	\item Aplicaciones: resistencia, deformación, diseño estructural.
%\end{itemize}

\subsection{Pasos para crear un estudio FEA}
% Ideal para mostrar con una pieza simple como una placa con carga.
%\begin{itemize}
%	\item Definir estudio estático.
%	\item Aplicar materiales, cargas y restricciones.
%	\item Mallado básico.
%\end{itemize}

\subsection{Interpretación de resultados}
% Muestra la tensión de Von Mises y deformación.
%\begin{itemize}
%	\item Resultados: desplazamientos, tensiones.
%	\item Factor de seguridad.
%\end{itemize}

\subsection{Consejos prácticos}
% Recomendaciones para evitar errores típicos como piezas flotantes o cargas mal definidas.
%\begin{itemize}
%	\item Validación de resultados.
%	\item Importancia de simplificar geometría y evitar contactos no definidos.
%\end{itemize}

\section{Animaciones en SolidWorks}

\subsection{Animaciones básicas}
% Ideal para mostrar presentaciones o ideas a clientes.
%\begin{itemize}
%	\item Uso del Motion Manager.
%	\item Creación de movimientos entre vistas o posiciones.
%\end{itemize}

\subsection{Animación de ensamblajes}
% Muy útil si modelan mecanismos. Mostrar cómo mover piezas, rotar engranajes, etc.
%\begin{itemize}
%	\item Relación con mates (coincidente, concéntrico).
%	\item Ejemplo: abrir una tapa, ensamblar un brazo robótico.
%\end{itemize}

\subsection{Exportar animaciones}
% Para presentaciones o portafolios.
%\begin{itemize}
%	\item Exportación a AVI o MP4.
%	\item Control de tiempo, cámara y calidad.
%\end{itemize}

\section{Recomendaciones para Profundizar}

%\subsection{Recursos adicionales}
% Aquí puedes recomendar canales de YouTube, cursos gratuitos o documentación oficial.
%\begin{itemize}
%	\item YouTube: JavelinTech, SolidWorks Tutorials, CAD CAM TUTORIAL.
%	\item Manual oficial de SolidWorks Simulation.
%\end{itemize}

%\subsection{Sugerencias prácticas}
% Ideas para proyectos donde se apliquen estos conceptos.
%\begin{itemize}
%	\item Diseñar una caja para electrónica y simular flujo de aire.
%	\item Simular una pieza estructural impresa en 3D.
%	\item Crear un mecanismo funcional con animación y FEA.
%\end{itemize}

