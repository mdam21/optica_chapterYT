\section{Introducción al Diseño CAD}

\subsection{Mentalidad de diseño CAD}
%\begin{itemize}
%	\item Pensar en términos de geometrías simples.
%	\item Enfoque práctico para abordar problemas de diseño.
%\end{itemize}

\subsection{Introducción a SolidWorks y software CAD similares}
%\begin{itemize}
%	\item SolidWorks, Fusion 360, Inventor, FreeCAD.
%	\item Importancia de conceptos comunes (paramétricos).
%\end{itemize}

\section{Principios Geométricos Básicos}

\subsection{Elementos básicos}
%\begin{itemize}
%	\item Puntos, líneas y planos.
%	\item Sistema de coordenadas cartesianas (X, Y, Z).
%\end{itemize}

\subsection{Ángulos y trigonometría básica}
%\begin{itemize}
%	\item Definiciones básicas de ángulos y triángulos.
%	\item Aplicación sencilla en CAD.
%\end{itemize}

\subsection{Restricciones geométricas y transformaciones}
%\begin{itemize}
%	\item Relaciones: paralelo, perpendicular, coincidente, etc.
%	\item Movimientos, rotaciones y escalado.
%\end{itemize}

\section{Interfaz Esencial de SolidWorks}

\subsection{Herramientas básicas de interfaz}
%\begin{itemize}
%	\item Barra de menú superior.
%	\item Command Manager.
%	\item Feature Manager Design Tree.
%	\item Área gráfica (Graphics Area).
%\end{itemize}

\subsection{Navegación básica del modelo}
%\begin{itemize}
%	\item Rotación y zoom con el mouse.
%	\item Vistas estándar (superior, frontal, lateral).
%\end{itemize}

\subsection{Árbol de diseño (Feature Manager)}
%\begin{itemize}
%	\item Historial de diseño y jerarquía de operaciones.
%	\item Operaciones padres e hijos.
%\end{itemize}

\section{Flujo de Trabajo CAD para Telescopios}

\subsection{Definición del problema de diseño}
%\begin{itemize}
%	\item Identificación de piezas clave (tubos, espejos, monturas).
%\end{itemize}

\subsection{Descomposición geométrica del telescopio}
%\begin{itemize}
%	\item Formas básicas: cilindros, discos, placas.
%\end{itemize}

\subsection{Construcción gradual del modelo}
%\begin{itemize}
%	\item Secuencia: boceto $\rightarrow$ operación 3D $\rightarrow$ ensamblaje.
%\end{itemize}

\section{Bocetos 2D (Sketches)}

\subsection{Conceptos iniciales}
%\begin{itemize}
%	\item Uso de planos predeterminados (Front, Top, Right).
%	\item Creación de nuevos planos.
%\end{itemize}

\subsection{Herramientas básicas de boceto}
%\begin{itemize}
%	\item Línea, círculo, rectángulo, polígono, spline.
%	\item Líneas constructivas.
%\end{itemize}

\subsection{Relaciones geométricas esenciales}
%\begin{itemize}
%	\item Coincidente, tangente, paralelo, igual, etc.
%\end{itemize}

\subsection{Cotas inteligentes}
%\begin{itemize}
%	\item Definir dimensiones y posiciones.
%	\item Importancia de bocetos completamente definidos.
%\end{itemize}

%\subsection{Práctica rápida}
%\begin{itemize}
%	\item Sección transversal de un tubo óptico.
%\end{itemize}

\section{Operaciones 3D Fundamentales para Telescopios}

\subsection{Extrusiones básicas}
%\begin{itemize}
%	\item Tubos, adaptadores y bases planas.
%\end{itemize}

\subsection{Revoluciones básicas}
%\begin{itemize}
%	\item Modelado de piezas cilíndricas y espejos parabólicos.
%\end{itemize}

\subsection{Operaciones de corte}
%\begin{itemize}
%	\item Cortes por extrusión y revolución.
%\end{itemize}

\subsection{Redondeos y chaflanes}
%\begin{itemize}
%	\item Suavizar aristas para impresión o montaje.
%\end{itemize}

\subsection{Patrones circulares y lineales}
%\begin{itemize}
%	\item Distribución simétrica de agujeros o soportes.
%\end{itemize}

%\subsection{Ejercicios prácticos}
%\begin{itemize}
%	\item Diseño de tubo para ocular.
%	\item Espejo primario básico por revolución.
%\end{itemize}

\section{Ensamblajes Básicos de Telescopios}

\subsection{Inserción y posicionamiento de piezas}
%\begin{itemize}
%	\item Insertar componentes.
%	\item Relaciones: concéntrica, coincidente, distancia.
%\end{itemize}

\subsection{Ejemplo de ensamblaje}
%\begin{itemize}
%	\item Ensamble simple: tubo + espejo + soporte.
%\end{itemize}

%\section{Dibujos Técnicos para Fabricación}


%\subsection{Creación de vistas técnicas}
%\begin{itemize}
%	\item Vistas frontal, lateral, superior, isométrica.
%\end{itemize}

\subsection{Acotado de piezas}
%\begin{itemize}
%	\item Uso de cotas inteligentes en vistas.
%\end{itemize}

\subsection{Exportación}
%\begin{itemize}
%	\item Exportar a PDF o plano físico.
%	\item Exportar bonito, para no perder los paths. 
%\end{itemize}

\section{Consejos para Fabricación con Impresión 3D}

\subsection{Preparación del modelo}
%\begin{itemize}
%	\item Inclinación, soporte, grosor mínimo.
%\end{itemize}

\subsection{Alternativas prácticas}
%\begin{itemize}
%	\item Uso de PVC, madera, corte láser.
%\end{itemize}

\subsection{Exportación a STL}
%\begin{itemize}
%	\item Preparar archivo para impresión 3D.
%\end{itemize}

\section{Preguntas y Revisión Final}
%
%\begin{itemize}
%	\item Espacio abierto para dudas.
%	\item Revisión rápida del contenido cubierto.
%\end{itemize}

%\section*{Anexo: Matemáticas en el Diseño}

%\subsection*{Matemáticas básicas-medias requeridas}
%\begin{itemize}
%	\item Bocetos 2D con cotas.
%	\item Operaciones de revolución (ángulos, radios, diámetros).
%	\item Distribuciones angulares (patrones circulares).
%\end{itemize}

%\subsection*{Posibles temas de clase avanzada}
%\begin{itemize}
%	\item Cálculo de parábolas para espejos.
%	\item Simulación óptica avanzada.
%	\item Modelado paramétrico con ecuaciones.
%	\item Análisis estructural (FEM).
%\end{itemize}