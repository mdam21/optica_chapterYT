\section*{Derivación de $c = \frac{1}{\sqrt{\varepsilon_0 \mu_0}}$ a partir de las ecuaciones de Maxwell}

Partimos de las ecuaciones de Maxwell en el vacío (sin cargas ni corrientes):

\begin{align}
	\nabla \cdot \vec{E} &= 0 \label{eq:gauss_electric}\\
	\nabla \cdot \vec{B} &= 0 \label{eq:gauss_magnetic}\\
	\nabla \times \vec{E} &= -\frac{\partial \vec{B}}{\partial t} \label{eq:faraday}\\
	\nabla \times \vec{B} &= \mu_0 \varepsilon_0 \frac{\partial \vec{E}}{\partial t} \label{eq:ampere_maxwell}
\end{align}

Aplicamos el operador rotacional a la ecuación de Faraday (\ref{eq:faraday}):

\begin{equation}
	\nabla \times (\nabla \times \vec{E}) = -\frac{\partial}{\partial t} (\nabla \times \vec{B}) \label{eq:rot_rot_E}
\end{equation}

Sustituyendo la ecuación de Ampère-Maxwell (\ref{eq:ampere_maxwell}) en (\ref{eq:rot_rot_E}):

\begin{equation}
	\nabla \times (\nabla \times \vec{E}) = -\mu_0 \varepsilon_0 \frac{\partial^2 \vec{E}}{\partial t^2} \label{eq:wave_raw}
\end{equation}

Utilizamos la identidad vectorial:
\[
\nabla \times (\nabla \times \vec{E}) = \nabla(\nabla \cdot \vec{E}) - \nabla^2 \vec{E}
\]

Dado que \( \nabla \cdot \vec{E} = 0 \), la identidad se reduce a:
\begin{equation}
	\nabla \times (\nabla \times \vec{E}) = -\nabla^2 \vec{E} \label{eq:vector_identity}
\end{equation}

Sustituyendo en la ecuación (\ref{eq:wave_raw}):

\begin{equation}
	-\nabla^2 \vec{E} = -\mu_0 \varepsilon_0 \frac{\partial^2 \vec{E}}{\partial t^2}
\end{equation}

Cancelamos los signos negativos:

\begin{equation}
	\nabla^2 \vec{E} = \mu_0 \varepsilon_0 \frac{\partial^2 \vec{E}}{\partial t^2} \label{eq:wave_equation}
\end{equation}

La ecuación (\ref{eq:wave_equation}) es la ecuación de onda para el campo eléctrico. Comparándola con la forma general de una ecuación de onda:

\[
\nabla^2 \vec{E} = \frac{1}{c^2} \frac{\partial^2 \vec{E}}{\partial t^2}
\]

Se concluye que:

\[
\frac{1}{c^2} = \mu_0 \varepsilon_0 \quad \Rightarrow \quad c = \frac{1}{\sqrt{\mu_0 \varepsilon_0}}
\]

\textbf{Por lo tanto, la velocidad de propagación de las ondas electromagnéticas en el vacío es:}

\[
c = \frac{1}{\sqrt{\varepsilon_0 \mu_0}}
\]
